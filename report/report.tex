\documentclass[12pt]{article}

\usepackage{sbc-template}

\usepackage{graphicx,url}

%\usepackage[brazil]{babel}
\usepackage[latin1]{inputenc}

\sloppy

\title{Lottery scheduling on xv6 with basic statistical analysis}

\author{Gabriel Batista Galli\inst{1}}


\address{Federal University of Fronteira Sul (UFFS)\\
  Mailbox 181 -- 89.802-112 -- Chapec� -- SC -- Brazil
  \email{g7.galli96@gmail.com}}

\begin{document}

\maketitle

\begin{abstract}
This paper describes an implementation of the lottery scheduler in the
xv6 operating system. Lottery scheduling is a process scheduling
mechanism that randomly chooses a process to run based on its tickets
(just like a lottery). The more tickets one process have, the higher the
probability of it being chosen by the scheduler. In the code presented
here, the Binary Indexed Tree (BIT, or Fenwick Tree) is used to
efficiently count every process' amount of tickets. Then we analyze the
scheduler behavior by looking at what order a group of processes with
the same task, but different amounts of tickets, finish their job.
\end{abstract}

\section{Introduction}
The xv6 operating system is a simple Unix-like teaching operating
system developed by MIT and widely used in Operating Systems classes all
over the world. Originally, every time the scheduler was to
choose a process to run, it would linearly search through the array of
processes looking for the first one that is \texttt{RUNNABLE} \cite{xv6book}. In
our class, we were assigned to implement the lottery scheduler on xv6.

As seen in \cite{waldspurger:94}, the lottery scheduler is a
probabilistic scheduler algorithm that works similarly to a real world
lottery. Every process is given a certain amount of tickets and the
scheduler randomly picks one of the available tickets among all
processes. The process holding the chosen ticket is then picked to be
run by the processor in the next quantum. This way, a process that has a
high quantity of tickets has a high probability of being chosen by the
scheduler and it will thus be run more frequently than a process that has less tickets.

\section{The lottery scheduler} \label{sec:lottsched}
In order to implement the lottery scheduler, a few changes were needed in
some data structures used by xv6 and a new library was created, \texttt{lottery.h}:

\begin{scriptsize}
\begin{verbatim}
#include "param.h"

#define MAXTICKS     NPROC * NPROC // maximum number of tickets a process is allowed to have
#define MINTICKS     1 // mininum number of tickets a process is allowed to have
#define SYSTICKS     MAXTICKS / 2 // number of tickets for all system processes
#define DEFTICKS     MAXTICKS / 2 // default number of tickets

#define min(a, b) ((a) < (b) ? (a) : (b))
#define max(a, b) ((a) > (b) ? (a) : (b))

unsigned long rand(void);
\end{verbatim}
\end{scriptsize}

The constants defined here are used to control the maximum and minimum
amount of tickets any process can hold. Additionally, the
\texttt{SYSTICKS} is the constant amount of tickets that every system
process receives and \texttt{DEFTICKS} is for the user if it does not
know how many tickets a process should have. The \texttt{rand} function was
already defined in the \texttt{usertests.c} file, but wasn't being used, so
it was moved to this library so it can be used when it is time to choose
the next process to run.

Then, in the \texttt{proc.h} file, the \texttt{tickets} attribute was
added to the structure \texttt{proc}, which describes a
process, to hold the quantity of tickets a process has:

\begin{scriptsize}
\begin{verbatim}
// Per-process state
struct proc {
  uint sz;                     // Size of process memory (bytes)
  pde_t* pgdir;                // Page table
  char *kstack;                // Bottom of kernel stack for this process
  enum procstate state;        // Process state
  int pid;                     // Process ID
  struct proc *parent;         // Parent process
  struct trapframe *tf;        // Trap frame for current syscall
  struct context *context;     // swtch() here to run process
  void *chan;                  // If non-zero, sleeping on chan
  int killed;                  // If non-zero, have been killed
  struct file *ofile[NOFILE];  // Open files
  struct inode *cwd;           // Current directory
  char name[16];               // Process name (debugging)
  int tickets;                 // Quantity of tickets assigned to this process
};
\end{verbatim}
\end{scriptsize}

Some other modifications made in the code relate to the \texttt{ptable}
structure, found in the \texttt{proc.c} file: the \texttt{deadstack}
array (a stack), its \texttt{top} and the \texttt{tickets} array:

\begin{scriptsize}
\begin{verbatim}
struct {
  int deadstack[NPROC], top;
  int tickets[NPROC + 1];
  struct spinlock lock;
  struct proc proc[NPROC];
} ptable;
\end{verbatim}
\end{scriptsize}

The \texttt{deadstack} array is a stack of the currently available
\texttt{pid}s. When xv6 starts and the function \texttt{pinit} is
called, \texttt{deadstack} will be initialized with all \texttt{pid}s
from NPROC to 1 (so the \texttt{pid} 1 will be at the top, and
the value of \texttt{top} will end up being equal to
\texttt{NPROC}). Then, whenever a new process is created, it will be
identified by the \texttt{pid} found on top of \texttt{deadstack}. Similarly, when a
process dies and its position in the \texttt{proc} array is once again set to
\texttt{UNUSED}, the \texttt{pid} that was being used by that process is
given back to the top of \texttt{deadstack} so it can be used
again later.

To keep track of every process' tickets, the \texttt{tickets} array was
created to be used as a Binary Indexed Tree (BIT, also known as Fenwick
Tree) to efficiently count how many tickets there are up until a given
process (among all existing processes). This way, we can binary search
our tree until we find the leftmost position in the array that
sums greater than or equal to the ticket that was picked this time
around. This position in the BIT is the chosen process \texttt{pid} and,
because of the \texttt{deadstack} way of handling \texttt{pid}s, it is
also its position in the \texttt{proc} array (but $- 1$, because
\texttt{proc} is indexed from 0 to $\texttt{NPROC} - 1$).

To manipulate the BIT, the functions \texttt{uptick} and
\texttt{ticount} are used. The former updates the BIT whenever a process is
created, executed or exits and the latter returns the sum of tickets up
until a given position. Just as \texttt{deadstack}, \texttt{tickets} is
initialized in the \texttt{pinit} function, where all positions are set
to 0.

This data structure was chosen because all operations (either getting the
sum or updating it) cost $O(\log n)$ each \cite{halim:13}. As an also
$O(\log n)$ binary search is used to find the chosen process, rather
than an $O(n)$ linear search, the total resulting complexity ends up being:

$$O(\log{(n \times \log n)}) = O(\log n + \log{(\log n)}) = O(\log n)$$

\subsection{Managing \texttt{pid}s}
The attribution of \texttt{pid}s is done in the \texttt{allocproc}
function, when creating a new process. As the \texttt{top} variable is
always one position above the last valid index of the stack, we
get it with \texttt{--ptable.top}. Likewise, we give a process' \texttt{pid} back
to the stack in the \texttt{wait} function, but this time with \texttt{ptable.top++}.

\subsection{Updating the BIT}
As said before, \texttt{uptick} is used to update the BIT whenever a
process is created, executed or exits. It is critical that we don't call
it any time more or less than needed, as the BIT will end up
accumulating a wrong amount of tickets or one process will be considered
to have more tickets than it actually does. Also note that passing a
positive argument to this function increments the BIT and passing a
negative argument decrements it. Incrementing is the act of giving
tickets to a process and decrementing is taking them away from it
(either permanently or temporarily, to avoid it being picked when it's
not \texttt{RUNNABLE}).

The BIT is updated in the functions that initialize a
process: incrementing on \texttt{userinit} and \texttt{fork}; when a
process is chosen to be executed: decrementing on \texttt{scheduler};
when a process gives up the processor because its time is over:
incrementing on \texttt{yield}; when a process is woken up: incrementing
on \texttt{wakeup1}, called by \texttt{wakeup}; and when a process is
killed and we have to wake up the parent, if it is sleeping, so it can
\texttt{wait} for that process to \texttt{exit}: incrementing, for that
parent, on \texttt{kill} (\texttt{wakeup} is not used here).

\subsection{Tickets}
As every process now needs tickets to be chosen to run and every process
should have an amount of tickets proportional to its importance or
priority, the user must be able to tell xv6 how many tickets its process
will need. To accomplish this, the \texttt{fork} system call was changed so that
it receives that amount of tickets as its sole argument.

Of course, the user might want to cheat and give the highest possible
amount of tickets to all of its processes so they run more frequently
than everyone else. Unfortunately, we can't do anything but trust in the
amount that we receive. The purpose of the lottery is simply run the
available processes proportionally to their amount of tickets. There is
no way to know how important a user process is. The only treatment done
here is to prevent a quantity less than the minimum or more than the
maximum (using the \texttt{min} and \texttt{max} macros). There is also the \texttt{DEFTICKS} constant that the user can
use, which is also automatically set if we receive 0 as argument.

\section{Statistical analysis and conclusion}
After all the modifications presented here, it is time to see if the
scheduler is working as intended. To test it, the
\texttt{lotterytests.c} file was created and configured as an xv6
executable just like \texttt{usertests.c}. It works like the following:
16 processes are created with the job of counting from 0 up to
$10^8$. The $i^{th}$ process received $i \times NPROC$ tickets. This
test was executed 100 times. A \texttt{cprintf} was purposely added to
the \texttt{exit} function, in the \texttt{proc.c} file, to print the
amount of tickets a process that just terminated had. This was later
processed in the host operating system by an ordinary C program that produced output conforming to the GNU Octave syntax to generate the following graphic showing the order in which those processes terminated:

\begin{figure}[h]
\includegraphics[scale=0.8]{tests-crop}
\centering
\end{figure}

Please note that the \emph{x-axis} represent the test number and the
\emph{y-axis} is the position between 1 and 16 that a process with a
number of tickets (represented by the color) terminated. Also take into
account that the processes were created sequentially from the one
holding the least amount of tickets to the one holding the highest
amount of tickets.

As a random algorithm, there is no
guarantee in the order of execution of the processes. The only sure
thing is that the higher the amount of tickets, the higher the
probability of being chosen to run and the larger the amount of tests,
the more this order will converge to the pattern established by that
probability. Additionally, there is no way to trigger the processes all at
once and this is also part of the reason that a process sometimes terminates
before another one that has more tickets: it started running before. If
the processes were created in the inverse order, from the one holding
the highest amount of tickets to the least amount of tickets, this
behavior would not happen so frequently.

Despite that, it is clear that the processes with the highest amount of
tickets terminated first (again, with some variation among
those with a similar quantity of tickets), and the ones with the lowest
amount of tickets terminated last, as we expected.

\bibliographystyle{sbc}
\bibliography{report}

\end{document}
